%%This is a very basic article template.
%%There is just one section and two subsections.
\documentclass[a4paper,12pt,technote]{IEEEtran}
% weitere Pakete
% Grafiken aus PNG Dateien einbinden
\usepackage{graphicx}

% figure refs with section number
\usepackage{chngcntr}
\counterwithin{figure}{section}

% deutsche Silbentrennung
%\usepackage[ngerman]{babel}

% Eurozeichen einbinden
\usepackage[right]{eurosym}

% Umlaute unter UTF8 nutzen
\usepackage[utf8]{inputenc}

% Zeichenencoding
\usepackage[T1]{fontenc}

\usepackage{lmodern}
\usepackage{fix-cm}

% floatende Bilder ermöglichen
%\usepackage{floatflt}

%todos can be marked
\usepackage{todonotes}
% mehrseitige Tabellen ermöglichen
\usepackage{longtable}

% Unterstützung für Schriftarten
%\newcommand{\changefont}[3]{ 
%\fontfamily{#1} \fontseries{#2} \fontshape{#3} \selectfont}

% Packet für Seitenrandabständex und Einstellung für Seitenränder
\usepackage{geometry}
\geometry{left=3.5cm, right=2cm, top=2.5cm, bottom=2cm}

% Paket für Boxen im Text
\usepackage{fancybox}


% bricht lange URLs "schoen" um
\usepackage[hyphens,obeyspaces,spaces]{url}

% Paket für Textfarben
\usepackage{color}

% Mathematische Symbole importieren
\usepackage{amssymb}

% auf jeder Seite eine Überschrift (alt, zentriert)
%\pagestyle{headings}

% erzeugt Inhaltsverzeichnis mit Querverweisen zu den Kapiteln (PDF Version)
%\usepackage[bookmarksnumbered,pdftitle={\titleDocument},hyperfootnotes=false,hidelinks]{hyperref}
%\usepackage[bookmarksnumbered,pdftitle={\titleDocument},hyperfootnotes=false]{hyperref}
%\hypersetup{colorlinks, citecolor=red, linkcolor=blue, urlcolor=black}
%\hypersetup{colorlinks, citecolor=black, linkcolor= black, urlcolor=black}

% neue Kopfzeilen mit fancypaket
\usepackage{fancyhdr} %Paket laden
\pagestyle{fancy} %eigener Seitenstil
\fancyhf{} %alle Kopf- und Fußzeilenfelder bereinigen
\fancyhead[L]{\nouppercase{\leftmark}} %Kopfzeile links
\fancyhead[C]{} %zentrierte Kopfzeile
\fancyhead[R]{\thepage} %Kopfzeile rechts
\renewcommand{\headrulewidth}{0.4pt} %obere Trennlinie
%\fancyfoot[C]{\thepage} %Seitennummer
%\renewcommand{\footrulewidth}{0.4pt} %untere Trennlinie

% für Tabellen
\usepackage{array}

% Runde Klammern für Zitate
%\usepackage[numbers,round]{natbib}

% Schaltet den zusätzlichen Zwischenraum ab, den LaTeX normalerweise nach einem Satzzeichen einfügt.
%\frenchspacing

% Paket für Zeilenabstand
\usepackage{setspace}

% für Bildbezeichner
\usepackage{capt-of}

% für Stichwortverzeichnis
\usepackage{makeidx}

% für Listings
\usepackage{listings}
%\lstset{
%  numbers=left,
%  numberstyle=\tiny,
%  numbersep=5pt,
%  breaklines=true,
%  breakatwhitespace,
%  keywordstyle=\color{black}\bfseries,
%  stringstyle=\sffamily,
%  showstringspaces=false,
%  basicstyle=\ttfamily\lst@ifdisplaystyle\footnotesize\else\normalsize\fi,
%  captionpos=b}
%\renewcommand{\lstlistingname}{Example}
%\AtBeginDocument{
%%\counterwithin{lstlisting}{section}
%}
%\setlength{\emergencystretch}{2pt}
% Indexerstellung
%\makeindex

% Abkürzungsverzeichnis
%\usepackage{nomencl}
%\let\abbrev\nomenclature

% Abkürzungsverzeichnis LiveTex Version
%\renewcommand{\nomname}{List of Abbreviations}
%\setlength{\nomlabelwidth}{.25\hsize}
%\renewcommand{\nomlabel}[1]{#1 \dotfill}
%\setlength{\nomitemsep}{-\parsep}
%\makenomenclature
%\makeglossary

% Abkürzungsverzeichnis TeTEX Version
% \usepackage[german]{nomencl}
% \makenomenclature
% %\makeglossary
% \renewcommand{\nomname}{Abkürzungsverzeichnis}
% \setlength{\nomlabelwidth}{.25\hsize}
% \renewcommand{\nomlabel}[1]{#1 \dotfill}
% \setlength{\nomitemsep}{-\parsep}

% Disable single lines at the start of a paragraph
%\clubpenalty = 10000
% Disable single lines at the end of a paragraph
%\widowpenalty = 10000
%\displaywidowpenalty = 10000

%\usepackage{titlesec}
%\subparagraph starting new line after heading
%\titleformat{\subparagraph}
%    {\normalfont\normalsize\bfseries}{\thesubparagraph}{1em}{}
%\titlespacing*{\subparagraph}{\parindent}{3.25ex plus 1ex minus .2ex}{.75ex plus .1ex}

% create additional subsection layer
%\makeatletter
%\renewcommand\paragraph{\@startsection{paragraph}{4}{\z@}
%	{-2.5ex\@plus -1ex \@minus -.25ex}
%	{1.25ex \@plus .25ex}
%	{\normalfont\normalsize\bfseries}}
%\makeatother
%\setcounter{secnumdepth}{4}
%\setcounter{tocdepth}{4}

\begin{document}
% hier werden die Trennvorschläge inkludiert
%\input{latex_einstellungen/trennung}

%syntax highlighting for additional languages
%\input{latex_einstellungen/systemverilog_code}
%\input{latex_einstellungen/e_code}

%Schriftart Helvetica
%\changefont{phv}{m}{n}

% Leere Seite am Anfang
%\newpage
%\thispagestyle{empty} % erzeugt Seite ohne Kopf- / Fusszeile
%\section*{ }

%use roman page numbers starting on this page
%\pagenumbering{roman}

% Titelseite %
%\include{latex_einstellungen/deckblatt}

% römische Numerierung
%\pagenumbering{arabic}

% 1.5 facher Zeilenabstand
%\onehalfspacing

% Sperrvermerk
%\input{sperrvermerk}
\title{Control and Status Register Files}
\author{Sebastian Wittka\\
Institute for Computer Engineering (ZITI) - Heidelberg University\\
Email: wittka@stud.uni-heidelberg.de}
\twocolumn[
\begin{@twocolumnfalse}
\maketitle
\end{@twocolumnfalse}]

% Einleitung / Abstract
\begin{abstract}
The steadily rising number of transistors on a chip increase the demand of generating functional units instead of designing everything manually. Therefore it is demonstated in this report how to fulfill this need for control and status register files. It is discussed why it is neccessary to generate such register files. The benefits and drawbacks of available generators are presented and especially the functionalities of the CAG Register File Generator are introduced.
\end{abstract}

% einfacher Zeilenabstand
%\singlespacing

% Eidesstattliche Erklärung
%\addcontentsline{toc}{section}{Declaration of Authorship}
%\include{erklaerung}

% Inhaltsverzeichnis anzeigen
%\newpage
%\tableofcontents
%\fancyhead[L]{}
% das Abbildungsverzeichnis
%\newpage
% Abbildungsverzeichnis soll im Inhaltsverzeichnis auftauchen
%\addcontentsline{toc}{section}{List of Figures}
% Abbildungsverzeichnis endgueltig anzeigen
%\listoffigures

% das Tabellenverzeichnis
%\newpage
% Abbildungsverzeichnis soll im Inhaltsverzeichnis auftauchen
%\addcontentsline{toc}{section}{Tabellenverzeichnis}
% \fancyhead[L]{Abbildungsverzeichnis / Abkürzungsverzeichnis} %Kopfzeile links
% Abbildungsverzeichnis endgueltig anzeigen
%\listoftables

%% WORKAROUND für Listings
%\makeatletter% --> De-TeX-FAQ
%\renewcommand*{\lstlistoflistings}{%
%  \begingroup
%    \if@twocolumn
%      \@restonecoltrue\onecolumn
%    \else
%      \@restonecolfalse
%    \fi
%    \lol@heading
%    \setlength{\parskip}{\z@}%
%    \setlength{\parindent}{\z@}%
%    \setlength{\parfillskip}{\z@ \@plus 1fil}%
%    \@starttoc{lol}%
%    \if@restonecol\twocolumn\fi
%  \endgroup
%}
%\makeatother% --> \makeatletter
% das Listingverzeichnis
%\newpage
% Listingverzeichnis soll im Inhaltsverzeichnis auftauchen
%\addcontentsline{toc}{section}{List of Examples}
%\fancyhead[L]{Abbildungs- / Tabellen- / Listingverzeichnis} %Kopfzeile links
%\renewcommand{\lstlistlistingname}{List of Examples}
%\lstlistoflistings
%%%%

% das Abkürzungsverzeichnis
%\newpage
% Abkürzungsverzeichnis soll im Inhaltsverzeichnis auftauchen
%\addcontentsline{toc}{section}{List of Abbreviations}
% das Abkürzungsverzeichnis entgültige Ausgeben
%\fancyhead[L]{Abkürzungsverzeichnis} %Kopfzeile links
%\input{latex_einstellungen/abkuezungen/abkuerzungen}
%\printnomenclature

% Definiert Stegbreite bei zweispaltigem Layout
%\setlength{\columnsep}{25pt}

%%%%%%% EINLEITUNG %%%%%%%%%%%%
%\twocolumn
%\newpage
%\fancyhead[L]{\nouppercase{\leftmark}} %Kopfzeile links

% 1,5 facher Zeilenabstand
%\onehalfspacing

%use arabic numbers starting with this page
%\pagenumbering{arabic}

% einzelne Kapitel
\section{Introduction}



%\include{2_uvm}

%\include{3_uvm_ml}

%\include{4_conclusion}


%\include{5_future_work}

%\include{beispiel}

%\onecolumn
% einfacher Zeilenabstand
%\singlespacing
% Literaturliste soll im Inhaltsverzeichnis auftauchen
%\addcontentsline{toc}{section}{References}
%\newpage

% Literaturverzeichnis anzeigen
%\renewcommand\refname{References}
%\bibliographystyle{plain}
%\bibliography{Seminar}
%% Index soll Stichwortverzeichnis heissen
% \newpage
% % Stichwortverzeichnis soll im Inhaltsverzeichnis auftauchen
% \addcontentsline{toc}{section}{Stichwortverzeichnis}
% \renewcommand{\indexname}{Stichwortverzeichnis}
% % Stichwortverzeichnis endgueltig anzeigen
% \printindex

%\onehalfspacing
% evtl. Anhang
%\newpage
%\addcontentsline{toc}{section}{Anhang}
%\fancyhead[L]{Anhang} %Kopfzeile links
%\input{anhang/anhang}

% leere Abschlussseite
%\newpage
%\thispagestyle{empty} % erzeugt Seite ohne Kopf- / Fusszeile
%\section*{ }


\end{document}
